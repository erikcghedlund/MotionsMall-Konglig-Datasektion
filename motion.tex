\documentclass[a4paper]{article}

\usepackage[default]{lato}
\usepackage[swedish]{babel}
\usepackage[dvipsnames]{xcolor}
\usepackage[T1]{fontenc}
\usepackage[utf8]{inputenc}
\usepackage{graphicx}
\usepackage{fancyhdr}
\usepackage{lastpage}
\usepackage{parskip}

\definecolor{cerise}{RGB}{232, 61, 132}
\definecolor{gray3}{RGB}{66, 66, 66}

\addtolength{\textheight}{4cm}
\addtolength{\voffset}{-2cm}
\addtolength{\textwidth}{2cm}
\addtolength{\hoffset}{-1cm}
\setlength{\headheight}{40pt}

\pagestyle{fancy}
\thispagestyle{fancy}
\fancyhead[L]{\small{
    Motion till Val-SM % Skriv SM-namn
\\\textcolor{gray3}{
    15-05-2023 % Datum för SM
}}}
\fancyhead[C]{\includegraphics[width=1.25cm,height=1.25cm]{./skold-color.pdf}}
\fancyhead[R]{\small{\textcolor{gray3}{
    Erik Hedlund\\ % Skriv ditt namn här
    sida \thepage{} av \pageref{LastPage}}}}
\fancyfoot[C]{\small{\textcolor{gray3}{Konglig Datasektionen, THS 100 44, datasektionen.se}}}
\renewcommand{\headrulewidth}{0pt}

\title{\textcolor{cerise}{\textbf{
    Motion angående\\
    Återkommande projekt i reglementet
}}}

\date{}

\begin{document}

\maketitle
\thispagestyle{fancy} % Smutsigt fulhack för att få header på första sidan

\section*{\textcolor{cerise}{
    Bakgrund
}}

Att återkommande projekt inte existerar i reglementet är lite jobbigt då det innebär att man
inte kan referera till dem samt reglera dem i sagda reglemente. En konsekvens av detta är till
exempel att projektledarposter inte kan vara intervjuposter då dessa poster inte existerar enligt
reglementet.

\section*{\textcolor{cerise}{
    Förslag till beslut
}}

	Mot bakgrund av ovanstående yrkar jag:

\newcounter{attnummer}
\setcounter{attnummer}{1}
\begin{list}{\bf att$_{\theattnummer}$\stepcounter{attnummer}}{}

% Basic saker

\item I reglementet tillägga \textit{\S4 Återkommande projekt} med text
    \begin{quote}
        Vald projektledare åläggs att inkomma med motion innehållandes budget samt verksamhetsplan för projektet till första möjliga SM efter valet, såvida SM inte redan beslutat om dessa för denna projektomgång.
    \end{quote}

% METAspexet

\item I reglementet tillägga \textit{\S4.1 METAspexet}
\item I reglementet tillägga \textit{\S4.1.1 Ändamål} med text
    \begin{quote}
    METAspexet är ett återkommande projekt som har delad verksamhet med motsvarande nämnd på Sektionen för Medieteknik

    METAspexet skall verka för att ge sektionsmedlemmar av Konglig Datasektionen och Sektionen för Medieteknik chans att utveckla flertalet färdigheter relaterad till teaterproduktion samt för att förstärka relationen mellan Konglig Datasektionen och Sektionen för Medieteknik.
    \end{quote}
\item I reglementet tillägga \textit{\S4.1.2 Organisation} med text
    \begin{quote}
        METAspexet leds av projektledare, benämnd Direqteur, för Konglig Datasektionen samt motsvarande nämndordförande
        för Sektionen för Medieteknik. Dessa projektledare och nämndordförande benämns \textit{METAspexets direqteurer}

        METAspexets direqteurer ansvarar för att rekrytera samt leda \textit{METAspexets chefsgrupp}

        METAspexet chefsgrupp ansvarar för att rekrytera samt leda METAspexet resterande grupper.
    \end{quote}
\item I reglementet tillägga \textit{\S4.1.3 Verksamhet} med text
    \begin{quote}
        \begin{itemize}
            \item Att sätta upp spexföreställningar under nästkommande kalenderår.
            \item Att ge medlemmar utrymme till engagemang och kreativt arbete samtidigt som de känner att de bidrar till ett större projekt.
            \item Att ge medlemmar möjlighet att lära sig nya saker tillsammans och av varandra.
            \item Att värna och utveckla relationen mellan Data och Media som sektioner, såväl som relationerna mellan sektionernas medlemmar.
        \end{itemize}
    \end{quote}
\item I reglementet tillägga \textit{\S4.1.4 Period} med text
    \begin{quote} Nya METAspexet-projekt öppnas årligen 1:a april och bör bedriva verksamhet framtill att dess alla föreställningar har skett nästkommande kalenderår.

    Projekttitel för respektive årgång är \textit{METAspexet <år>}, där <år> är det år METAspexets föreställningar planeras hållas.
    \end{quote}

% dÅre

\item I reglementet tillägga \textit{\S4.2 dåre}
\item I reglementet tillägga \textit{\S4.2.1 Ändamål} med text
    \begin{quote}
        dÅre är ett återkommande projekt som syftar att främja idrottsengagemanget inom sektionen samt främja gemenskap mellan studenter från olika årgångar genom att organisera en skidsemester.
    \end{quote}
\item I reglementet tillägga \textit{\S4.2.2 Organisation} med text
    \begin{quote}
        dÅre leds av projektledare

        Projektledare för dÅre ansvarar för att rekrytera samt leda dÅres projektgrupp, benämnd \textit{dÅrestaben}.
    \end{quote}
\item I reglementet tillägga \textit{\S4.2.3 Verksamhet} med text
    \begin{quote}
        \begin{itemize}
            \item Att organisera en skidsemester för Konglig Datasektionens medlemmar.
            \item Att uppmuntra medlemmar oberoende av tidigare erfarenhet att delta i alpint idrottsengagemanget.
        \end{itemize}
    \end{quote}
\item I reglementet tillägga \textit{\S4.2.4 Period} med text
    \begin{quote}
        Nya dåre-projekt öppnas årligen 1:a april och bör bedriva verksamhet framtill att dess planerade skidresa skett nästkommande kalenderår.

        Projekttitel för respektive årgång är \textit{dÅre <år>}, där <år> är det år dÅres skidresa planeras hållas.
    \end{quote}

% Studs

\item I reglementet tillägga \textit{\S4.3 Studs}
\item I reglementet tillägga \textit{\S4.3.1 Ändamål} med text \begin{quote} TODO \end{quote}
\item I reglementet tillägga \textit{\S4.3.2 Organisation} med text
    \begin{quote}
        Studs leds av projektledare

        Två projektledare bör väljas

        Projektledare för Studs ansvarar för att rekrytera samt leda Studs projektgrupp
    \end{quote}
\item I reglementet tillägga \textit{\S4.3.3 Verksamhet} med text
    \begin{quote}
        TODO
    \end{quote}
\item I reglementet tillägga \textit{\S4.3.4 Period} med text
    \begin{quote}
        Nya Studs-projekt öppnas årligen 1:a april och bör bedriva verksamhet framtill att dess studieresa har skett nästkommande kalenderår.

        Projekttitel för respektive årgång är \textit{Studs <år>}, där <år> är det år Studs resa planeras hållas.
    \end{quote}

% Vårbalen

\item I reglementet tillägga \textit{\S4.4 Vårbalen}
\item I reglementet tillägga \textit{\S4.4.1 Ändamål} med text
    \begin{quote}
        TODO
    \end{quote}
\item I reglementet tillägga \textit{\S4.4.2 Organisation} med text
    \begin{quote} Vårbalen leds av projektledare, benämnd \textit{Vårbalsgeneral}
        Vårbalsgeneral ansvarar för att rekrytera samt leda Vårbalens projektgrupp, benämnd \textit{Vårbalsarmen}
    \end{quote}
\item I reglementet tillägga \textit{\S4.4.3 Verksamhet} med text
    \begin{quote}
        TODO
    \end{quote}
\item I reglementet tillägga \textit{\S4.4.4 Period} med text
    \begin{quote}
        Nya Vårbalen-projekt öppnas årligen 1:a april och bör bedriva verksamhet framtill att dess planerade sittning skett nästkommande kalenderår.
        
        Projekttitel för respektive årgång är \textit{Vårbalen <år>}, där <år> är det år Vårbalens sittning planeras hållas.
    \end{quote}

% Project pride

\item I reglementet tillägga \textit{\S4.5 Project pride}
\item I reglementet tillägga \textit{\S4.5.1 Ändamål} med text
    \begin{quote}
        TODO
    \end{quote}
\item I reglementet tillägga \textit{\S4.5.2 Organisation} med text
    \begin{quote}
        Project Pride leds av projektledare

        Projektledare för Project Pride ansvarar för att rekrytera samt leda projektgruppen för Project Pride.
    \end{quote}
\item I reglementet tillägga \textit{\S4.5.3 Verksamhet} med text
    \begin{quote}
        TODO
    \end{quote}
\item I reglementet tillägga \textit{\S4.5.4 Period} med text
    \begin{quote}
        Nya Project Pride-projekt öppnas årligen 16:e november och bör bedriva verksamhet framtill att dess ??? har skett nästkommande kalenderår.

        Projekttitel för respektive årgång är \textit{Project Pride <år>}, där <år> är det år Project Prides ??? planeras hållas.
    \end{quote}

% dJubileet
%
% \item I reglementet tillägga \textit{\S4.6 dJubileet}
% \item I reglementet tillägga \textit{\S4.6.1 Ändamål} med text \begin{quote} TODO \end{quote}
% \item I reglementet tillägga \textit{\S4.6.2 Organisation} med text \begin{quote} dJubileet leds av projektledare, benämnd \textit{Jubileumsmarskalk}\\\\Jubileumsmarskalk ansvarar för att rekrytera samt leda dJubileets projektgrupp\end{quote}
% \item I reglementet tillägga \textit{\S4.6.3 Verksamhet} med text \begin{quote} TODO \end{quote}
% \item I reglementet tillägga \textit{\S4.6.4 Period} med text \begin{quote} Nya dJubileet-projekt öppnas vart femte år, räknat från 1986, 16:e november och bör bedriva verksamhet framtill att dess event har skett nästkommande jubileumsår.\\\\Projekttitel för respektive årgång är \textit{dJubileet <år>}, där <år> är det år dJubileets event planeras hållas.\end{quote}

% Funktionärer

\item I reglementet tillägga \textit{\S5.3 Projektledare för återkommande projekt}
\item I reglementet tillägga \textit{\S5.3.1 Direqteur} med text \begin{quote} Är projektledare för METAspexet \end{quote}
\item I reglementet tillägga \textit{\S5.3.2 Projektledare för dÅre}
\item I reglementet tillägga \textit{\S5.3.4 Vårbalsgeneral} med text \begin{quote} Är projektledare för Vårbalen \end{quote}
\item I reglementet tillägga \textit{\S5.3.5 Projektledare för Project Pride}
\item I reglementet tillägga \textit{\S5.3.6 Jubileumsmarskalk} med text \begin{quote} Är projektledare för dJubileet \end{quote}

% Stadgeändring

\item I stadgarna ändra \textit{\S7.2 Uppstart av återkommande projekt} från \begin{quote} Projekt som D-rektoratet anser är regelbundet återkommande kan startas utan SM-beslut. De startas då istället genom beslut på DM med en kort motivering innehållande referens till minst ett väldigt likartat tidigare projekt samt en uppskattning av de uppgifter, bortsett från budget, som krävs för att starta ett nytt projekt. Val av projektledare ska ske på nästkommande SM enligt samma procedur som för övriga funktionärsposter. Vald projektledare åläggs att inkomma med motion innehållandes budget samt verksamhetsplan för projektet till första möjliga SM efter valet, såvida SM inte redan beslutat om dessa för denna projektomgång. D-rektoratet ansvarar för att en förteckning över återkommande projekt finns tillgänglig för alla sektionens medlemmar på dess hemsida. \end{quote} till \begin{quote} Projekt skrivna som återkommande under reglementet startas utan beslut på bestämt datum. Val av projektledare ska ske på nästkommande SM enligt samma procedur som för övriga funktionärsposter. Om D-rektoratet anser att ett återkommande projekt inte ska öppnas för nästa verksamhetsperiod så kan de bestämma det på ett DM innan datumet som projektet öppnas. \end{quote}

% Fippel

\item Konsekvensnumrera alla punkter jag fuckar upp genom att peta in Återkommande projekt under \S4.
\end{list}

\vspace{2cm}
\noindent
% Name, eventuell titel eller annat skoj
Erik Hedlund, Projektledare för ett återkommande projekt

% Avkommentera följande om du vill visa en bild
% \begin{center}
% \includegraphics[scale=0.5]{
    % Eventuell fin bild du vill visa.
%     ./logo.png
% }
% \end{center}


\end{document}
